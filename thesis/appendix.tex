\chapter{\label{chap:bib_an}Bibliography analysis}

\begin{figure*}[!h]
\begin{center}
\includegraphics[width=130mm]{img/prototype/biblio-screenshot.png}
\caption{The cloud is made up of the top-$50$ weighted terms, according to the tf-idf term weighting scheme (cf.\ section~\ref{subsec:tfidf}), of the centroid (cf.\ equation~\ref{eq:cl_centroid}) of the term-by-document matrix (cf.\ equation~\ref{eq:term_doc_matrix}) built by relying upon the vector space model of the words within the abstract section of the papers from bibliography.}
\end{center}
\end{figure*}

\begin{figure*}
\begin{center}
\includegraphics[width=140mm]{img/prototype/leo.pdf}
\caption{PageRank analysis, performed with a perturbation coefficient $\alpha=0.85$, of the citation graph induced by citations of papers from the bibliography (cf.\ section~\ref{sec:link_analysis}). It highlights the main topics which have been addressed in this work, i.e.\ 1) automatic information extraction from scientific literature; 2) mining models while analyzing scientific literature; 3) \textsc{p2p} distributed content location.}
\end{center}
\end{figure*}

\begin{figure*}
\begin{center}
\includegraphics[width=140mm]{img/prototype/biblio_lsi_3d.pdf}
\caption{Projection of bibliography document vectors into a 3-d semantic subspace obtained by retaining the $3$ largest singular values while performing LSI (cf.\ section~\ref{sec:LSI})}
\end{center}
\end{figure*}
