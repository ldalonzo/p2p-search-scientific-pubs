Peer-to-peer (p2p) distributed computing model is based on the sharing of resources between peers which leads to the design and deployment of very large applications with very low cost.

Content location is a general problem in p2p networks. Solutions which rely on broadcast are inherently not scalable, while those based on DHTs lead to high overhead while requiring a keyword-based approach to information retrieval. Most of p2p systems delegates the task of finding relevant content to an information retrieval system based on a client-server paradigm. This approach requires sizeable capital investments in the infrastructure in order to achieve high scalability levels and it is prone to possible data exploitation.

Recent solutions have proposed to organize network topology around data semantics and to exploit the resulting topology in order to facilitate content location. Tribler is a p2p file sharing system which forms a semantic overlay network to identify related content, facilitate search and recommend content to user without the need of a central service.

By relying on this latter model we designed a proof-of-concept fully-decentralized search and recommender system tailored for scientific literature. We proactively manage a social network which aggregates users with similar information needs by relying on a lightweight gossip-based topology management protocol. User's information needs are automatically learned by tracking users reading habits. A statistical model of language, such as the vector space model of text, is used in order to capture the semantics of user's interests. Content location and recommendation are carried out by exploiting the dynamically evolving social network. Relevant information is expected to be located in the user's neighborhood since the network topology has been constructed around user's information needs, i.e.\ it has been pushed towards relevant information. Interesting paper are recommended by users with similar reading interests.
